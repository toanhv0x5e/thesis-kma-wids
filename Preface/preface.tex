\chapter*{MỞ ĐẦU}
\addcontentsline{toc}{chapter}{MỞ ĐẦU}
%\begin{preface}
Theo một báo cáo đầu năm 2017~\cite{kemp2017digital}, Việt Nam hiện có hơn 49 triệu người dùng Internet, là một trong những quốc gia có tỷ lệ người dùng Internet cao nhất trên thế giới. Trong đó, đa phần là người dùng sử dụng các thiết bị thông minh như máy tính xách tay, điện thoại thông minh, máy tính bảng và Internet TV. Cùng với sự phát triển của Internet, mạng WiFi cũng dần thay thế cho mạng có dây truyền thống. Mạng WiFi có mặt ở khắp mọi nơi, từ các nơi công cộng như sân bay, khách sạn, quán cà phê đến các hộ gia đình. Mạng WiFi mang lại những lợi ích rõ rệt, nó giúp cho việc truyền tải, tiếp nhận thông tin cực kỳ nhanh chóng và tiện lợi, giúp người sử dụng công nghệ tiết kiệm thời gian, nâng cao hiệu quả công việc~\cite{Khan2017}.

Bên cạnh những lợi ích như vậy, mạng WiFi vẫn tồn tại những vấn đề khiến người dùng lo lắng, như ảnh hưởng của sóng WiFi đối với sức khỏe con người (mặc dù chưa có bằng chứng khoa học nào chính thức đề cập~\cite{foster2013wi}), hay nguy cơ mất an toàn thông tin trong mạng. Nguy cơ mất an toàn thông tin trong mạng WiFi hiện đang là vấn đề rất cấp thiết, cần có sự phối hợp giải quyết của cả tổ chức quản lý hạ tầng mạng và người dùng. Hiện nay trên thế giới đã có nhiều giải pháp giúp phát hiện sớm các cuộc tấn công trong mạng WiFi, tạo ra các cảnh báo tới nhà quản trị và thậm chí có thể ngăn chặn cuộc tấn công đang diễn ra~\cite{karen2015comparing}. Tuy nhiên, hầu hết chúng đều là các giải pháp dành cho doanh nghiệp lớn, cần chi phí đầu tư lớn cho một hạ tầng mạng WiFi hoàn chỉnh.

Với ý tưởng phục vụ cho đối tượng là doanh nghiệp nhỏ, quán cà phê và hộ gia đình, đồ án này sẽ nghiên cứu các giải pháp sẵn có, từ đó xây dựng một hệ thống phát hiện xâm nhập mạng WiFi dựa trên các phần mềm mã nguồn mở. Hệ thống này có thể tích hợp dễ dàng lên hạ tầng thiết bị thông thường, người dùng có thể sở hữu một hệ thống phát hiện xâm nhập mạng WiFi với các tính năng an toàn, giao diện quản lý thân thiện, chỉ với chi phí rất thấp.\\

Do thời gian nghiên cứu còn hạn chế, báo cáo này không thể tránh khỏi một số sai sót, khuyết điểm. Tác giả kính mong quý Thầy, Cô và các bạn đóng góp ý kiến để đồ án hoàn thiện và thực tiễn hơn. Xin chân thành cảm ơn!

\begin{flushright}
    Tp. Hồ~Chí~Minh, ngày 04 tháng 05 năm 2017

    {\textbf{Sinh viên thực hiện}\hspace*{1.25cm}\par}
    \vspace*{1cm}
    \textbf{Hà~Văn~Toàn}\hspace*{2cm}
\end{flushright}
%\end{preface}
